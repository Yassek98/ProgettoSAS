\documentclass[12pt]{article} % invece del default 10pt
\usepackage{graphicx} % Required for inserting images
\usepackage[italian]{babel}
\usepackage[hidelinks]{hyperref} % Nasconde i link rossi
\usepackage{xcolor} % Per i colori
\usepackage[top=2cm, bottom=2cm, left=2cm, right=2cm]{geometry} % Margini ridotti
\usepackage{multicol} % Per le colonne
\usepackage[table]{xcolor}
\usepackage[svgnames]{xcolor} % Nel preambolo
\usepackage{array}
\makeatletter
\newcommand{\arraybackslash}{\let\\=\@arraycr}
\makeatother
\usepackage{float} % Per [H] che forza la posizione
\usepackage{rotating} % Necessario per l'ambiente sidewaysfigure
\usepackage[utf8]{inputenc}
\usepackage{amsmath}
\usepackage{amssymb}
\usepackage{enumitem}


\title{\Huge\bfseries\color[rgb]{0,0.4,0}Sviluppo e applicazioni software}
\author{Veronica Bosso 1043229, Yassine Sadek 1051617
}
\date{Anno accademico 2024/2025}

\begin{document}

\maketitle

\tableofcontents
\newpage

\section{Requisiti}
\subsection{Use Case Dettagliato (\textbf{UC})}
\vspace{1.7\baselineskip}

\textbf{GESTIRE IL PERSONALE}\mbox{}\\[2mm]
Gestire il personale significa inserire i dati dei lavoratori e eventualmente modificarli/eliminarli. Significa cioè gestire altri dati rilevanti che riguardano la vita del personale dentro l’azienda per es. ferie, carriera ...

\vspace{2\baselineskip}

\noindent\textbf{\large\color[rgb]{0,0.4,0}Analisi dell'User Stories}\mbox{}

% Due elenchi affiancati in colonne
\begin{multicols}{2}

% Primo elenco (colonna sinistra)
\textbf{Robert}
\begin{enumerate}
    \item Ha un elenco di persone che può contattare via e-mail o whatsapp, per gestire il personale degli eventi
    \item Riceve segnalazioni per nuovo personale: segna nome e contatto
    \item Completa le informazioni (indirizzo e cod. fiscale) al primo ingaggio
    \item Opzionalmente, consulta lo storico degli eventi e si confronta con gli event manager per valutare un'\textbf{assunzione} permanente.
    \item Riceve e valuta le richieste di \textbf{ferie} in base alla disponibilità e agli eventi pianificati.
    \item Se la richiesta è accettata, aggiorna le informazioni del personale.
\end{enumerate}

\columnbreak % Forza il passaggio alla colonna successiva

% Secondo elenco (colonna destra)
\textbf{Raffaele}
\begin{enumerate}
    \item Controlla le disponibilità del personale (prima quello permanente)
    \item Se rimangono scoperti dei ruoli li cerca nell’elenco dei collaboratori occasionali
    \item Se qualcuno non è più disponibile lo cancella dall’elenco, o aggiorna le informazioni se cambiano
    \item Prende nota delle performance del personale per future valutazioni in caso di possibilità di assunzione ``permanente''.
\end{enumerate}

\end{multicols}

% Spazio prima delle conclusioni
\vspace{1cm}

% Sezione conclusioni
\noindent\textbf{Processo combinato: Robert (proprietario, organizzatore) e Raffaele (organizzatore)}

\begin{enumerate}
    \item Consulta l'elenco del personale\footnote{Unisce le azioni iniziali di entrambi gli organizzatori.}.

    \item Seleziona un membro del personale dall'elenco.\\
    \textbf{OPPURE} Aggiunge una nuova persona all'elenco\footnote{Solo il proprietario può eseguire questa azione.}, inserendo le informazioni preliminari (nome, contatto).\\
    \textit{Se aggiunge una nuova persona, termina il caso d'uso.}
    
    \item \textbf{OPZIONALMENTE} decide di modificare le informazioni della persona selezionata (es. dati di contatto, indirizzo, codice fiscale al primo ingaggio). \textit{Termina il caso d'uso.}\\[2mm]
    \textbf{OPPURE} decide di eliminare la persona dall'elenco del personale, in caso non sia più disponibile\footnote{Il sistema imposta lo stato a 'inattivo' invece di cancellare fisicamente.}. \textit{Termina il caso d'uso.}\\[2mm]
    \textbf{OPPURE} consulta lo storico degli eventi e le note sulle performance per valutarne l'operato.\\
    \textit{Se decide di promuovere il membro selezionato da ''occasionale'' a ''permanente''\footnote{Azione eseguibile solo dal proprietario, consultando gli organizzatori interessati.}, esegue la promozione. Termina il caso d'uso.}\\[2mm]
    \textbf{OPPURE}\footnote{Azione eseguibile solo dal proprietario.} valuta la richiesta di ferie incrociando il monte ferie residuo con gli eventi pianificati.\\
    \textit{Se decide di accettare la richiesta di ferie, modifica lo stato del personale selezionato da ''Disponibile'' a ''In ferie'' per il periodo richiesto. Termina il caso d'uso.}
\end{enumerate}
\vspace{2\baselineskip}

\noindent\textbf{\large\color[rgb]{0,0.4,0}Informazioni generali}\mbox{}\\[2mm]
\textbf{Nome caso d'uso}: Gestire il personale.\\
\textbf{Portata}: Sistema gestionale del personale.\\
\textbf{Livello}: Obiettivo utente.\\
\textbf{Attore primario}: Organizzatore, proprietario.\\
\textbf{Parti Interessate}: Proprietario, organizzatori, personale.\\
\textbf{Pre-condizioni}: L'attore è autenticato nel sistema gestionale come Organizzatore, per alcuni operazioni l'attore deve essere autenticato come Proprietario.\\
\textbf{Garanzie di successo o post-condizioni}: Le modifiche ai dati del personale sono salvate in modo persistente e tracciate. L'elenco del personale è aggiornato.
\vspace{0.3cm}

\subsubsection{Scenario Principale di successo}

\begin{center}
\renewcommand{\arraystretch}{1.5}
\rowcolors{1}{}{}
\begin{tabular}{|c|>{\raggedright\arraybackslash}p{0.45\textwidth}|>{\raggedright\arraybackslash}p{0.45\textwidth}|}
\hline
\rowcolor{ForestGreen!40}
\textbf{\#} & \multicolumn{1}{|c|}{\textbf{Attore}} & \multicolumn{1}{|c|}{\textbf{Sistema}} \\
\hline
    1 & Consulta l'elenco del personale.&   Fornisce l'elenco completo del personale, con opzioni di ricerca e filtro\\\hline
    2& Seleziona un membro del personale dall'elenco.	&Fornisce i dati e il profilo completo della persona selezionata.\\\hline

 3& Opzionalmente decide di modificare le informazioni della persona selezionata (es. dati di contatto, indirizzo).& Salva le modifiche apportate e conferma l'avvenuto aggiornamento.\\\hline
  & \textit{Termina il caso d’uso} &\\ \hline
\end{tabular}
\end{center}

\newpage

\subsubsection{Estensioni}
\vspace{1.5\baselineskip}

\noindent\textbf{\large\color[rgb]{0,0.4,0} Estensione 2a - \textbf{Aggiunta personale}}
\begin{center}
\renewcommand{\arraystretch}{1.5}
\rowcolors{1}{}{}
\begin{tabular}{|c|>{\raggedright\arraybackslash}p{0.45\textwidth}|>{\raggedright\arraybackslash}p{0.45\textwidth}|}
\hline
\rowcolor{ForestGreen!40}
\textbf{\#} & \multicolumn{1}{|c|}{\textbf{Attore}} & \multicolumn{1}{|c|}{\textbf{Sistema}} \\
\hline
   2a.1& Seleziona l'opzione per aggiungere una nuova persona.&   Presenta un modulo vuoto per
l’inserimento dei dati del nuovo membro del personale.\\\hline
 2a.2& Inserisce le informazioni preliminari (nome, contatto) e conferma.&Salva i dati inseriti e aggiunge una nuova persona all'elenco.\\\hline
 & \textit{Termina il caso d’uso}&\\ \hline
\end{tabular}
\end{center}
\vspace{1.5\baselineskip}

\noindent\textbf{\large\color[rgb]{0,0.4,0} Estensione 3a - Eliminazione personale}
\begin{center}
\renewcommand{\arraystretch}{1.5}
\rowcolors{1}{}{}
\begin{tabular}{|c|>{\raggedright\arraybackslash}p{0.45\textwidth}|>{\raggedright\arraybackslash}p{0.45\textwidth}|}
\hline
\rowcolor{ForestGreen!40}
\textbf{\#} & \multicolumn{1}{|c|}{\textbf{Attore}} & \multicolumn{1}{|c|}{\textbf{Sistema}} \\
\hline
   3a.1& Decide di eliminare dall'elenco la persona selezionata.&   Imposta lo stato del collaboratore a 'inattivo' e lo rimuove dell'elenco principale.\\\hline
 & \textit{Termina il caso d’uso}&\\\hline
\end{tabular}
\end{center}
\vspace{1.5\baselineskip}

\noindent\textbf{\large\color[rgb]{0,0.4,0} Estensione 3b - Gestione carriera (Promozione)}
\begin{center}
\renewcommand{\arraystretch}{1.5}
\rowcolors{1}{}{}
\begin{tabular}{|c|>{\raggedright\arraybackslash}p{0.45\textwidth}|>{\raggedright\arraybackslash}p{0.45\textwidth}|}
\hline
\rowcolor{ForestGreen!40}
\textbf{\#} & \multicolumn{1}{|c|}{\textbf{Attore}} & \multicolumn{1}{|c|}{\textbf{Sistema}} \\
\hline
   3b.1& Consulta lo storico
degli eventi e le note sulle performance per
valutarne l’operato.&   Fornisce lo storico degli eventi a cui il
collaboratore ha partecipato e le relative
note di valutazione.\\\hline
 3b.2& Opzionalmente modifica lo stato del personale selezionato
da ”occasionale” a ”permanente”.&Aggiorna lo stato contrattuale del
collaboratore e registra la promozione.\\\hline
 & \textit{Termina il caso d’uso}&\\ \hline
\end{tabular}
\end{center}
\newpage

\noindent\textbf{\large\color[rgb]{0,0.4,0} Estensione 3c - Gestione richiesta di ferie}
\begin{center}
\renewcommand{\arraystretch}{1.5}
\rowcolors{1}{}{}
\begin{tabular}{|c|>{\raggedright\arraybackslash}p{0.45\textwidth}|>{\raggedright\arraybackslash}p{0.45\textwidth}|}
\hline
\rowcolor{ForestGreen!40}
\textbf{\#} & \multicolumn{1}{|c|}{\textbf{Attore}} & \multicolumn{1}{|c|}{\textbf{Sistema}} \\\hline
 3c.1& Seleziona una richiesta di ferie pendente per il membro del personale&Mostra il periodo delle ferie richieste.\\\hline

   3c.2& Valuta la richiesta di
ferie incrociando il monte ferie residuo con
gli eventi pianificati.&   Fornisce i dati di supporto alla decisione:
monte ferie residuo del collaboratore e
calendario che evidenzia i conflitti con i
turni già assegnati.\\\hline
 3c.3& Opzionalmente modifica lo stato del personale selezionato
da ”Disponibile” a ”In ferie” per il periodo
di ferie richiesto.&Aggiorna la disponibilità del collaboratore,
e scala dal monte ferie il periodo di ferie
che è stato accettato.\\\hline
 & \textit{Termina il caso d’uso}&\\ \hline
\end{tabular}
\end{center}
\vspace{1.5\baselineskip}

\subsubsection{Eccezioni}
\vspace{1.5\baselineskip}

\noindent\textbf{\large\color{red} Eccezione 2a.1a - Permessi non sufficienti}
\begin{center}
\renewcommand{\arraystretch}{1.5}
\rowcolors{1}{}{}
\begin{tabular}{|c|>{\raggedright\arraybackslash}p{0.45\textwidth}|>{\raggedright\arraybackslash}p{0.45\textwidth}|}
\hline
\rowcolor{ForestGreen!40}
\textbf{\#} & \multicolumn{1}{|c|}{\textbf{Attore}} & \multicolumn{1}{|c|}{\textbf{Sistema}} \\
\hline
   2a.1a.1& Seleziona l'opzione per aggiungere una nuova persona.&   Mostra messaggio di errore: l'attore che sta cercando di eseguire questa operazione non ha permessi sufficienti (solo il proprietario può aggiungere personale).\\\hline
 & \textit{Termina il caso d’uso}&\\\hline
\end{tabular}
\end{center}
\vspace{1.5\baselineskip}

\noindent\textbf{\large\color{red} Eccezione 2a.2a - \textbf{Contatto duplicato} }
\begin{center}
    \renewcommand{\arraystretch}{1.5}
    \begin{tabular}{|c|>{\raggedright\arraybackslash}p{0.45\textwidth}|>{\raggedright\arraybackslash}p{0.45\textwidth}|}
        \hline
        \rowcolor{ForestGreen!40}
        \textbf{\#} & \multicolumn{1}{|c|}{\textbf{Attore}} & \multicolumn{1}{|c|}{\textbf{Sistema}} \\
        \hline
        2a.2a.1& Inserisce nome e contatto di un nuovo collaboratore.&
        Il sistema rileva che esiste già un collaboratore attivo con lo stesso contatto (numero di telefono).
        Segnala l'errore e richiede di usare un contatto diverso.\\
        \hline
        & \textit{Il caso d'uso ritorna al passo 2a.1}&\\
        \hline
    \end{tabular}
\end{center}
\newpage

\noindent\textbf{\large\color{red} Eccezione 3a.1a - \textbf{Impossibile eliminare: collaboratore con incarichi futuri} }
\begin{center}
\renewcommand{\arraystretch}{1.5}
\rowcolors{1}{}{}
\begin{tabular}{|c|>{\raggedright\arraybackslash}p{0.45\textwidth}|>{\raggedright\arraybackslash}p{0.45\textwidth}|}
\hline
\rowcolor{ForestGreen!40}
\textbf{\#} & \multicolumn{1}{|c|}{\textbf{Attore}} & \multicolumn{1}{|c|}{\textbf{Sistema}} \\
\hline
   3a.1a.1& 		Decide di eliminare dall’elenco la persona
selezionata.&   Blocca l'eliminazione. Comunica l'errore e mostra l'elenco degli incarichi futuri che impediscono l'operazione.
\\\hline
 & \textit{Termina il caso d’uso}&\\\hline
\end{tabular}
\end{center}
\vspace{2\baselineskip}

\noindent\textbf{\large\color{red} Eccezione 3b.2a - \textbf{Permessi non sufficienti} }
\begin{center}
\renewcommand{\arraystretch}{1.5}
\rowcolors{1}{}{}
\begin{tabular}{|c|>{\raggedright\arraybackslash}p{0.45\textwidth}|>{\raggedright\arraybackslash}p{0.45\textwidth}|}
\hline
\rowcolor{ForestGreen!40}
\textbf{\#} & \multicolumn{1}{|c|}{\textbf{Attore}} & \multicolumn{1}{|c|}{\textbf{Sistema}} \\
\hline
   3b.2a.1& 		Modifica lo stato del personale selezionato da ”occasionale” a ”permanente”.&   Mostra messaggio di errore: l’attore che sta cercando di eseguire questa
operazione non ha permessi sufficienti (solo
il proprietario può promuovere il personale).\\\hline
 & \textit{Termina il caso d’uso}&\\\hline
\end{tabular}
\end{center}
\vspace{2\baselineskip}

\noindent\textbf{\large\color{red} Eccezione 3c.3a - \textbf{Permessi non sufficienti} }
\begin{center}
\renewcommand{\arraystretch}{1.5}
\rowcolors{1}{}{}
\begin{tabular}{|c|>{\raggedright\arraybackslash}p{0.45\textwidth}|>{\raggedright\arraybackslash}p{0.45\textwidth}|}
\hline
\rowcolor{ForestGreen!40}
\textbf{\#} & \multicolumn{1}{|c|}{\textbf{Attore}} & \multicolumn{1}{|c|}{\textbf{Sistema}} \\
\hline
    3c.3a.1& 		Modifica lo stato del personale selezionato
da ”Disponibile” a ”In ferie” per il periodo
di ferie richiesto.&   Mostra messaggio di errore: l’attore che sta cercando di eseguire questa
operazione non ha permessi sufficienti (solo
il proprietario può accettare una richiesta di ferie).\\\hline
 & \textit{Termina il caso d’uso}&\\\hline
\end{tabular}
\end{center}
\vspace{1.5\baselineskip}

\noindent\textbf{\large\color{red} Eccezione 3c.3b - \textbf{Numero ferie superato} }
\begin{center}
\renewcommand{\arraystretch}{1.5}
\rowcolors{1}{}{}
\begin{tabular}{|c|>{\raggedright\arraybackslash}p{0.45\textwidth}|>{\raggedright\arraybackslash}p{0.45\textwidth}|}
\hline
\rowcolor{ForestGreen!40}
\textbf{\#} & \multicolumn{1}{|c|}{\textbf{Attore}} & \multicolumn{1}{|c|}{\textbf{Sistema}} \\
\hline
    3c.3b.1& 		Modifica lo stato del personale selezionato
da ”Disponibile” a ”In ferie” per il periodo
di ferie richiesto.&   Non può eseguire l'azione perchè il monte ferie non copre la richiesta di ferie. Segnale l'errore.\\\hline
 & \textit{Termina il caso d’uso}&\\\hline
\end{tabular}
\end{center}
\vspace{1.5\baselineskip}

\noindent\textbf{\large\color{red} Eccezione 3c.3c - \textbf{Ferie sovrapposte} }
\begin{center}
    \renewcommand{\arraystretch}{1.5}
    \begin{tabular}{|c|>{\raggedright\arraybackslash}p{0.45\textwidth}|>{\raggedright\arraybackslash}p{0.45\textwidth}|}
        \hline
        \rowcolor{ForestGreen!40}
        \textbf{\#} & \multicolumn{1}{|c|}{\textbf{Attore}} & \multicolumn{1}{|c|}{\textbf{Sistema}} \\
        \hline
        3c.3c.1& Approva la richiesta di ferie.&
        Il sistema rileva che il periodo richiesto si sovrappone a ferie già approvate per lo stesso collaboratore.
        Blocca l'approvazione e segnala il conflitto.\\
        \hline
        & \textit{Termina il caso d'uso}&\\
        \hline
    \end{tabular}
\end{center}

\newpage

\subsection{\textbf{Modello di Dominio}}
% Inserire qui il modello di dominio
\begin{figure}[H]
    \centering
    \hspace*{-1.2cm} % Sposta a sinistra di 0.5cm
    \includegraphics[width=1.15\linewidth, height=0.9\textheight, keepaspectratio]{MD_3UC-U3.drawio.png}
    \caption{Modello di Dominio dopo l'aggiunta di gestione del personale}
    \label{fig:dominio}
\end{figure}
\subsubsection{Regole di Business}
% Inserire qui le regole di business
\begin{itemize}
    \item Una voce puo' comparire in un  menu' all'interno di una sezione oppure come voce "libera" quindi  verrà valorizzata  una e una sola  delle due  relazioni appartiene
    \item Il Cuoco che esegue il Compito  deve essere disponibile  nel Turno in cui lo svolge 
    \item Se un compito ha completato = sì allora  il compito deve avere obbligatoriamente  associato un cuoco ed un turno. In alternativa,  tale assegnamento non è obbligatorio.
\end{itemize}
\newpage

\subsection{System Sequence Diagram (\textbf{SSD})}
\vspace{1.5\baselineskip}
\subsubsection{Scenario Principale di successo}
% Inserire qui gli SSD


% ===================================================================
% FIGURA 1: SCENARIO PRINCIPALE DI SUCCESSO
% Immagine: input_file_0.png
% ===================================================================
\begin{figure}[H]
    \centering
    % La riga \hspace sposta l'immagine a sinistra. 
    % Potrebbe non essere necessaria per questa immagine più stretta.
    % Puoi commentarla (mettendo un % all'inizio) o rimuoverla per centrarla normalmente.
    \includegraphics[width=1.0\linewidth, keepaspectratio]{input_file_0.png}
    \caption{Diagramma di Sequenza: Scenario Principale di Successo (Modifica di un collaboratore).}
    \label{fig:ssd-successo}
\end{figure}

\subsubsection{Estensioni ed Eccezioni}
% ===================================================================
% FIGURA 2: FLUSSO DI AGGIUNTA DI UN NUOVO COLLABORATORE
% Immagine: input_file_2.png
% ===================================================================
\begin{figure}[H]
    \centering
    % Questi valori potrebbero necessitare di aggiustamenti per adattarsi alla tua pagina
    \hspace*{-1cm} 
    \includegraphics[width=0.8\linewidth, keepaspectratio]{input_file_2.png}
    \caption{Diagramma di Sequenza: Flusso di Aggiunta di un Nuovo Collaboratore \\
    (Estensione 2a).}
    \label{fig:ssd-aggiunta}
\end{figure}


% ===================================================================
% FIGURA 3: AZIONI AVANZATE SUL COLLABORATORE 
% ===================================================================
\begin{figure}[H]
    \centering
    % Rimuovi completamente \hspace* per evitare problemi di allineamento
    % Usiamo \textwidth che è la larghezza effettiva della pagina (escludendo i margini).
    \includegraphics[width=0.7\textwidth]{input_file_1.png}
    \caption{Diagramma di Sequenza: Azioni Avanzate sul Collaboratore (Estensioni 3a, 3b, 3c).}
    \label{fig:ssd-avanzate}
\end{figure}

\subsection{\textbf{Contratti delle Operazioni}}
\vspace{1.5\baselineskip}

\textbf{Pre-condizione generale}: l'attore è identificato con un'istanza \texttt{\textit{org}} di \texttt{Collaboratore} e \texttt{\textit{org}} \textbf{ricopre} un \texttt{Ruolo} con i permessi necessari.

\subsubsection*{Operazioni di Interrogazione}
Queste operazioni recuperano dati senza modificare lo stato del sistema e, pertanto, non hanno post-condizioni.

\begin{enumerate}
    \item {\fontseries{b}\selectfont\large visualizzaElencoPersonale()}\\
    \textbf{Pre-condizioni:}
    \begin{itemize}[leftmargin=*,noitemsep,topsep=-5pt]
        \item[-] nessuna oltre a quella generale
    \end{itemize}
    \textbf{Post-condizioni:}
     \begin{itemize}[leftmargin=*,noitemsep,topsep=-5pt]
        \item[-] nessuna
    \end{itemize}
    
    \item {\fontseries{b}\selectfont\large visualizzaProfiloCompleto(\underline{collab}: Collaboratore)}\\
    \textbf{Pre-condizioni:}
    \begin{itemize}[leftmargin=*,noitemsep,topsep=-5pt]
        \item[-] nessuna oltre a quella generale
    \end{itemize}
    \textbf{Post-condizioni:}
     \begin{itemize}[leftmargin=*,noitemsep,topsep=-5pt]
        \item[-] nessuna
    \end{itemize}

    \item[3b.1] {\fontseries{b}\selectfont\large visualizzaStoricoPerformance(\underline{collab}: Collaboratore)}\\
    \textbf{Pre-condizioni:}
    \begin{itemize}[leftmargin=*,noitemsep,topsep=-5pt]
        \item[-] è in corso la visualizzazione del profilo di \texttt{\textit{collab}}
    \end{itemize}
    \textbf{Post-condizioni:}
     \begin{itemize}[leftmargin=*,noitemsep,topsep=-5pt]
        \item[-] nessuna
    \end{itemize}

    \item[3c.1] {\fontseries{b}\selectfont\large visualizzaRichiesteFerie(\underline{collab}: Collaboratore)}\\
    \textbf{Pre-condizioni:}
    \begin{itemize}[leftmargin=*,noitemsep,topsep=-5pt]
        \item[-] è in corso la visualizzazione del profilo di \texttt{\textit{collab}}
    \end{itemize}
    \textbf{Post-condizioni:}
     \begin{itemize}[leftmargin=*,noitemsep,topsep=-5pt]
        \item[-] nessuna
    \end{itemize}

    \item[3c.2] {\fontseries{b}\selectfont\large valutaRichiestaFerie(\underline{rich}: RichiestaFerie)}\\
    \textbf{Pre-condizioni:}
    \begin{itemize}[leftmargin=*,noitemsep,topsep=-5pt]
        \item[-] è in corso la visualizzazione delle richieste ferie di un collaboratore
        \item[-] \texttt{\textit{rich}} è \textbf{avanzata da} quel collaboratore
    \end{itemize}
    \textbf{Post-condizioni:}
     \begin{itemize}[leftmargin=*,noitemsep,topsep=-5pt]
        \item[-] nessuna
    \end{itemize}
\end{enumerate}
\newpage

\subsubsection*{Operazioni di Modifica}

\begin{enumerate}
    \item[2a.1] {\fontseries{b}\selectfont\large iniziaAggiuntaCollaboratore()}\\
    \textbf{Pre-condizioni:}
    \begin{itemize}[leftmargin=*,noitemsep,topsep=-5pt]
        \item[-] \texttt{\textit{org}} \textbf{ricopre} un \texttt{Ruolo} con \texttt{tipo = 'Proprietario'}.
    \end{itemize}
    \textbf{Post-condizioni:}
    \begin{itemize}[leftmargin=*,noitemsep,topsep=-5pt]
        \item[-] È stata creata un'istanza \texttt{\textit{collab}} di \texttt{Collaboratore} (in memoria, non ancora persistita)
        \item[-] È in corso l'aggiunta del nuovo \texttt{Collaboratore} \texttt{\textit{collab}}
    \end{itemize}
    \textit{Nota: la persistenza avviene solo dopo la conferma tramite \texttt{aggiungiCollaboratore()}.}

    \item[2a.2] {\fontseries{b}\selectfont\large aggiungiCollaboratore(\underline{nome}: testo, \underline{contatto}: numero)}\\
    \textbf{Pre-condizioni:}
    \begin{itemize}[leftmargin=*,noitemsep,topsep=-5pt]
        \item[-] è in corso l'aggiunta del nuovo \texttt{Collaboratore} \texttt{\textit{collab}}
    \end{itemize}
    \textbf{Post-condizioni:}
    \begin{itemize}[leftmargin=*,noitemsep,topsep=-5pt]
        \item[-] \texttt{\textit{collab}.nome = \underline{nome}}
        \item[-] \texttt{\textit{collab}.contatto = \underline{contatto}}
        \item[-] \texttt{\textit{collab}.occasionale = 'si'}
        \item[-] \texttt{\textit{collab}.attivo = 'si'}
    \end{itemize}

    \item[3.] {\fontseries{b}\selectfont\large modificaInfoProfilo(\underline{nuovoNome?}: testo, \underline{nuovoCodFisc?}: testo,\\ \underline{nuovoContatto?}: numero, \underline{nuovoIndirizzo?}: testo)}\\
    \textbf{Pre-condizioni:}
    \begin{itemize}[leftmargin=*,noitemsep,topsep=-5pt]
        \item[-] è in corso la visualizzazione del profilo di \texttt{\textit{collab}}
    \end{itemize}
    \textbf{Post-condizioni:}
    \begin{itemize}[leftmargin=*,noitemsep,topsep=-5pt]
        \item[-] [se specificato] \texttt{\textit{collab}.nome = \underline{nuovoNome}}
        \item[-] [se specificato] \texttt{\textit{collab}.codFisc = \underline{nuovoCodFisc}}
        \item[-] [se specificato] \texttt{\textit{collab}.contatto = \underline{nuovoContatto}}
        \item[-] [se specificato] \texttt{\textit{collab}.indirizzo = \underline{nuovoIndirizzo}}
    \end{itemize}

    \item[3a.1] {\fontseries{b}\selectfont\large eliminaCollaboratore()}\\% Nota: qui si assume che collab sia noto dal contesto "in corso"
    \textbf{Pre-condizioni:}
    \begin{itemize}[leftmargin=*,noitemsep,topsep=-5pt]
        \item[-] è in corso la visualizzazione del profilo di \texttt{\textit{collab}}
        \item[-] non esiste un'istanza \texttt{\textit{a}} di \texttt{CollaboratorAvailability} tale che
        \texttt{\textit{a}.collaborator = \textit{collab}} e
        \texttt{\textit{a}.confirmed = 'si'} e \texttt{\textit{a}.shift.data $>$ data\_corrente}    \end{itemize}
    \textbf{Post-condizioni:}
    \begin{itemize}[leftmargin=*,noitemsep,topsep=-5pt]
        \item[-] \texttt{\textit{collab}.attivo = 'no'}
    \end{itemize}

    \item[3b.2] {\fontseries{b}\selectfont\large promuoviCollaboratore()}\\
    \textbf{Pre-condizioni:}
    \begin{itemize}[leftmargin=*,noitemsep,topsep=-5pt]
        \item[-] è in corso la visualizzazione del profilo di \texttt{\textit{collab}}
        \item[-] \texttt{\textit{org}} \textbf{ricopre} un \texttt{Ruolo} con \texttt{tipo = 'Proprietario'}
        \item[-] \texttt{\textit{collab}.occasionale = 'si'}
    \end{itemize}
    \textbf{Post-condizioni:}
    \begin{itemize}[leftmargin=*,noitemsep,topsep=-5pt]
        \item[-] \texttt{\textit{collab}.occasionale = 'no'}
    \end{itemize}
    \newpage

    \item[3c.3] {\fontseries{b}\selectfont\large aggiornaStatoFerie(\underline{rich}: RichiestaFerie, \underline{approvata}: si/no)}\\
    \textbf{Pre-condizioni:}
    \begin{itemize}[leftmargin=*,noitemsep,topsep=-5pt]
        \item[-] \texttt{\textit{org}} \textbf{ricopre} un \texttt{Ruolo} con \texttt{tipo = 'Proprietario'}
        \item[-] sia \texttt{\textit{collab}} il \texttt{Collaboratore} tale che \texttt{\textit{rich}} è \textbf{avanzata da} \texttt{\textit{collab}}
        \item[-] \texttt{\textit{rich}.approvata = 'no'} (o comunque è pendente)
        \item[-] [se \underline{approvata} = 'si'] \texttt{\textit{collab}.monteFerie} $\geq$ (\texttt{\textit{rich}.dataFine - \textit{rich}.dataInizio})
    \end{itemize}
    \textbf{Post-condizioni:}
    \begin{itemize}[leftmargin=*,noitemsep,topsep=-5pt]
        \item[-] \texttt{\textit{rich}.approvata = \underline{approvata}}
        \item[-] [se \underline{approvata} = 'si']
        \begin{itemize}
            \item[-] \texttt{\textit{rich}.approvata = 'si'} (il periodo è confermato)
            \item[-] Per ogni data \texttt{d} tale che
            \texttt{\textit{rich}.dataInizio $\leq$ d $\leq$ \textit{rich}.dataFine}:
            \texttt{\textit{collab}} risulta \textbf{in ferie} nella data \texttt{d}.
            \item[-] \texttt{\textit{collab}.monteFerie = \textit{collab}.monteFerie - durata}
        \end{itemize}
    \end{itemize}
\end{enumerate}
\newpage

\section{Progettazione}
\subsection{Design Class Diagram (\textbf{DCD})}
% Inserire qui il DCD completo per UC1, UC2 e UC3
\begin{figure}[H]
    \centering
    \hspace*{-1.7cm}
    \includegraphics[width=1.2\linewidth, keepaspectratio]{DCD definitivo.jpg}
    \caption{DCD copre: Gestione dei menù, gestione dei compiti della cucine e \textbf{gestione del personale}}
    \label{fig:ssd-successo}
\end{figure}

\subsection{Design Sequence Diagram (\textbf{DSD})}
\subsubsection{DSD 1 - [addCollaborator]}
% Descrizione breve
\begin{figure} [H]
    \centering
    \hspace*{-1.8cm}
    \includegraphics[width=1.2\linewidth]{addCollaborator.png}
    \caption{DSD \textbf{addCollaborator(name, contact)}}
    \label{fig:DSD 1}
\end{figure}
\newpage

\subsubsection{DSD 2 - [updateCollaboratorInfo]}
% Descrizione breve
\begin{figure}[H]
    \centering
    \hspace*{-1.8cm}
    \includegraphics[width=1.2\linewidth]{updateCollaboratorInfo.png}
    \caption{DSD \textbf{updateCollaboratorInfo(collab, ...)}}
    \label{fig:DSD 2}
\end{figure}
\newpage

\subsubsection{DSD 3 - [removeCollaborator]}
% Descrizione breve
\begin{figure}[H]
    \centering
    \hspace*{-1.8cm}
    \includegraphics[width=1.2\linewidth]{removeCollaborator.png}
    \caption{DSD \textbf{removeCollaborator(collab)}}
    \label{fig:DSD 3}
\end{figure}
\newpage

\subsubsection{DSD 4 - [evaluateLeaveRequest]}
% Descrizione breve
\begin{figure}[H]
    \centering
    \hspace*{-1.8cm}
    \includegraphics[width=1.2\linewidth]{evaluateLeaveRequest}
    \caption{DSD \textbf{evaluateLeaveRequest(req, approved)}}
    \label{fig:DSD 4}
\end{figure}
\newpage

\subsubsection{DSD 5 - [promoteCollaborator]}
% Descrizione breve
\begin{figure}[H]
    \centering
    \hspace*{-1.8cm}
    \includegraphics[width=1.2\linewidth]{promoteCollaborator.png}
    \caption{DSD \textbf{promoteCollaborator(collab)}}
    \label{fig:DSD 5}
\end{figure}
\newpage

\subsubsection{DSD 6 - [logPerformance]}
% Descrizione breve
\begin{figure}[H]
    \centering
    \hspace*{-1.8cm}
    \includegraphics[width=1.2\linewidth]{logPerformance.png}
    \caption{DSD \textbf{logPerformance(collab, event, text)}}
    \label{fig:DSD 6}
\end{figure}

\end{document}